%% LyX 2.2.1 created this file.  For more info, see http://www.lyx.org/.
%% Do not edit unless you really know what you are doing.
\documentclass{article}
\usepackage[T1]{fontenc}
\usepackage[latin9]{inputenc}
\setcounter{secnumdepth}{4}
\setlength{\parindent}{0bp}

\makeatletter
%%%%%%%%%%%%%%%%%%%%%%%%%%%%%% User specified LaTeX commands.
\renewcommand{\partname}{Article}

\makeatletter
\@addtoreset{section}{part}
\makeatother

\makeatother

\begin{document}

\title{Charter for Cybersecurity at the University of California, Riverside
(Cyber@UCR)}

\date{Version of March 10th, 2017}

\maketitle
\pagebreak{}

\tableofcontents{}

\pagebreak{}

\part{Organizational Identity}

\section{Organizational Name}

The organization will be formally known as Cybersecurity at the University
of California, Riverside. It may be referred to in shorthand as \textquotedbl{}Cyber@UCR.\textquotedbl{}

\section{Organizational Purpose}

The mission of Cyber@UCR is the advancement of cybersecurity education
at the University of California, Riverside through: - Organizing teams
to represent UCR at cybersecurity competitions - Advocating for the
use of secure design principles in engineering - Raising awareness
of topics related to information security - Promoting the professional
development of students interested in security-related careers - Forming
faculty and community partnerships for the advancement of information
security awareness - Advancing the ethical study of cybersecurity

\section{Historical Acknowledgements}

Cyber@UCR was formed as a joint project between chapters of the Institute
for Electrical and Electronics Engineers (IEEE) and Association for
Computing Machinery (ACM) at the University of California, Riverside
(UCR). This charter incorporates the original joint committee for
cybersecurity between those two organizations into a single club.

\part{Membership}

Membership shall be extended to any student of the University of California,
Riverside. Members will fall into three general categories: General
Members, Alumni Members, and Faculty / Staff Advisor.

\section{Nondiscrimination Clause}

Active membership of this student organization shall be chosen without 
discrimination on the basis of race, religion, sex, sexual orientation, color,
disability, national origin, age, or marital status, except in cases of fraternity
and sorority organization which are exempt by federal law from Title IX
regulations concerning discrimination on the basis of sex. Active membership 
should also be composed of at least 50\% UCR Students (undergraduate or 
graduate).

\section{General Members}

A General Member is any registered member who is an enrolled, active
student at UCR. This may be any undergraduate or graduate student
currently pursuing an academic credential at UCR.

\section{Alumni Members}

An Alumni Member is any registered member who is an alumnus of the
University of California, Riverside. These members may not vote in
General Elections, nor be elected as Officers. They are to be welcome,
where appropriate, at all gatherings of the organization, and may
be invited to share personal and professional experiences as speakers. 

\section{Faculty and Staff Advisors}

A Faculty or Staff Advisor is any member of UCR's faculty or staff
who is registered with the organization and provides instructional
or logistical support.

\section{Founders}

Founding members of Cyber@UCR, who helped initially build the organization,
are accorded special status within the membership. These are defined
as the original eleven members of the 2016 Western Regional Collegiate
Cyber-Defense Competition team, and a listing of those names is provided
as an addendum.

\section{Definition of \textquotedbl{}Registered.\textquotedbl{}}

The Club Secretary will maintain the Roster of registered members,
and will decide the registration status of individual participants.
Standards to maintain registration, such as payment of dues or attendence
requirements, must be agreed upon by a majority of the Board and applied
equally to each person in a membership category.

\part{Officers}

The club's Board shall consist of these officers
\begin{itemize}
\item President
\item Vice President
\item Secretary
\item Treasurer
\item Outreach Director
\item Ethics Director
\item Team / Project Leaders
\item Graduate Student Advisors
\item Faculty / Staff Advisors
\end{itemize}

\section{Requirements for Service}

All officers shall be in good academic standing (not on academic probation,
etc). All officers shall be General Members.

Members who hold an officer position in another campus organization
shall not be eligible to serve as President, Vice President, Secretary,
or Treasurer.

No provision of this Charter categorally bars graduate students from
holding an officer position. However, where possible, undergraduate
candidates should be preferred for these positions, with graudate
students fulfilling an advisory role. In the absence of qualified
undergraduate candidates, a graduate student may be elected through
the normal process.

\section{The President}

The President is the executive head of the organization, and will
be chiefly responsible for defining the organization's priorities
and means for achieving the organization's objectives, as described
by the Organizational Purpose in this Charter.

The President shall: 
\begin{itemize}
\item Nominate leaders for cybersecurity competitions. These leaders will
be confirmed by a vote of the Board.
\item Act as the spokesperson for the organization.
\item Preside over meetings of the Board and General Members.
\item Be one of three signers on any financial documents.
\item Be the principal organizer for any conferences or meetings.
\item Be the tiebreaker in any tied votes, and vote only in the case of
a tie.
\item Performs or delegates any task not covered in this Charter.
\end{itemize}

\section{The Vice President}

The Vice President directly assists the President in the execution
of his or her duties, and performs the President's duties in his or
her absence.

The Vice President shall: 
\begin{itemize}
\item Carry out the duties of the President in his or her absence. 
\item Assume the role of the President in the event of resignation or other
extended vacancy, until a new President is elected.
\item Assist the President in his or her duties.
\end{itemize}

\section{The Secretary}

The Secretary is the organization's chief historian and record-keeper,
maintaining documents and completing administrative tasks critical
to the efficient operation of the Organization.

The Secretary shall: 
\begin{itemize}
\item Maintain the Membership Roster.
\item Maintain the organizational email address.
\item Record minutes at meetings.
\item Record the outcome of elections and voting.
\item Act as the organizational historian.
\item Secure meeting places.
\item Maintain organizational files and oversees administrative affairs.
\end{itemize}

\section{The Outreach Director}

The Outreach Director is the organization's chief recruiter, and helps
grow the membership of the organization by coordinating outreach events.
This position also helps retention by planning social events.

The Outreach Director shall: 
\begin{itemize}
\item Arrange social events.
\item Coordinate tabling for the organization during club rush / fairs.
\item Maintain the organization's public web presence.
\item Be principally responsible for the organization's growth and recruiting
efforts.
\end{itemize}

\section{The Ethics Director}

The Ethics Director and the Ethics Committee he or she oversees is
the Organization's chief oversight. Cyber@UCR teaches skills that
could easily be employed harmfully, and the Ethics Director is the
first line of defense against member misconduct. As such, the Ethics
Director is empowered and encouraged to question the ethics of decisions
by the Board, and to vigorously combat unethical behavior by the organization's
members. The Ethics Director should be an individual of high moral
character.

The Ethics Director shall:
\begin{itemize}
\item Enforce the organizational Code of Ethics.
\item Preside over ethics hearings.
\item Recommend the dismissal of members who violate organizational and
professional ethics.
\item Refer violations of campus policy or applicable local, state, or federal
law to the appropriate authorities.
\item Educate members on the importance of and standards for ethical behavior.
\item Act as ethical oversight for the rest of the organizations activities.
\item Nominate members of an Ethics Committee, to number no more than one
committee member per twenty-five active general members.
\end{itemize}

\section{Competitive Team / Project Leaders}

Competitive Team / Project Leaders are those people who lead Cyber@UCR
santioned teams of students in cybersecurity competitions, or who
lead special projects (such as organizing a specific, major hackathon).
These Leaders may also hold other Officer positions. Those who are
not already Officers are made members of the Board and afforded all
the board voting rights of an officer. They may have more descriptive
titles that better describe their function (such as \textquotedbl{}hackathon
coordinator\textquotedbl{}). These positions last as only long as
the project or competition they are associated with. If the project
or competition rolls over into a new academic year, the appointee
must be re-confirmed at the first board meeting of each new academic
year.

Competitive Team / Project Leaders shall: 
\begin{itemize}
\item Be a General Member
\item Be responsible for outreach and advertisement of their specific Project
or Competition.
\item Manage volunteers or competitiors participating in the Project or
Competition.
\item Ensure all appropriate paperwork, funding, and other administrative
work associated with the Project or Competition is completed in a
timely fashion.
\item Where the event is a Competition, be appraised of all rules regarding
the event.
\item Where the event is a Competition, secure training.
\end{itemize}

\section{Graduate Student Advisor }

The Graduate Student Advisor is not a voting member of the Board,
but is present to act in an advisory role. This is an informal role,
and the Board may retain any number of Advisors as is deemed situationally
appropriate.

\section{Treasurer}

The Treasurer shall:
\begin{itemize}
\item Collect dues. 
\item Maintain an accounting of the organization's budget and funding. 
\item Maintain financial accounts. 
\item Acts as a signer on financial documents.
\item Maintain an inventory of all equipment held by the organization, and
periodically verify the accuracy of the inventory.
\end{itemize}

\section{The Ethics Committee}

The Ethics Committee shall consist of the Ethics Director plus an
additional member per twenty-five registered general members. Members
of the Ethics Committee are nominated by the Ethics Director and confirmed
by a majority vote of the Board. Ethics Committee members who are
not the Ethics Director may not vote in Board Elections. The Ethics
Committee may consist exclusively of the Ethics Director where membership
is small.

\section{Unfilled Positions}

\subsection{Normal Vacancies}

With the exception of ethical functions (the Ethics Director and the
Ethics Committee), where an officer position is vacant, the President
may delegate those functions to existing Officers or perform them
on their own. Ethical functions must always remain independent of
the rest of the Executive.

Where the Ethics Director position is vacant, a nominee must be selected
by General Election at the earliest possible opportunity.

\subsection{Removal of Officers}

Officers may be removed by an Ethics Hearing, as described in Article
V, or by a vote of no confidence by a supermajority in a General Election.
A motion for a vote of no confidence may be initiated by any General
Member and must be seconded by another General Member.

Officers will be given one week notice prior to a no-confidence vote,
and will be allowed to address the General Members before a vote is
taken.

A Special Election of the general members will be called to
elect a replacement officer within two weeks of the officer's date
of departure from office.

\subsection{Resignation}

Officers must give two weeks notice of their intention to resign,
and then may resign their office. A replacement Officer may then be
selected according to the requirements of that position on the effective
date of their resignation, via the normal nomination and general election
process. A Special Election of the general members will be called to
elect a replacement officer within two weeks of the officer's date
of departure from office.

\subsection{Departing Officers}

Officers who depart must turn over any applicable records, keys, access
codes, or other materials a successor should require. Departing Officers
should meet with successors for at least one hour to facilitate a
turnover of responsiblities. Incoming Officers must immediately change
access codes or update any other access control mechanisms to prevent
access to those systems by a predecessor.

\section{Terms of Office, Nominations, Officer Elections}

Officers serve for one academic year. During the Spring Quarter of
each academic year, the General Members will hold a General Election
to select the next academic year's officers. The new Officers begin
their terms on the final day of instruction of each Spring Quarter.

Any General Member in good academic standing and who has not been
sanctioned by an Ethics Hearing in the previous year may announce
their candidacy for an Officer position.

Any Officer may serve in that same office for two academic years.

\section{Faculty Advisors}

The Faculty Advisors are faculty or staff members of the University
of California, Riverside, who mentor the organization. This position
does not have a term limitation. If the organization is without a
Faculty Advisor, a new Faculty Advisor must be selected and approved
by a simple majority vote of the Board within fourteen days.

Faculty Advisors do not participate in organizational elections.

A Faculty Advisor may be removed by a supermajority vote of the General
Members. The Faculty Advisor must be notified one week in advance
of a vote for removal, and must be given an opportunity to address
the General Members prior to voting.

\part{Elections}

\section{Majorities and Quorums}

A simple majority is a conensus of at least 51\% of eligible voters.

A supermajority is a consensus of at least two thirds (67\%) of eligible
voters.

An election is only valid if an appropriate quorum is established.
A quorum is defined as: For General Elections, at least one third
of the registered membership. For Board Elections, at least three
fourths of eligible officers. Founders who would not otherwise be
eligible to vote are not counted when determining if a quorum exists. 

\section{General Elections}

General Elections are votes on issues cast by the entire General Membership.
Eligible voters are those who are recorded as General Members by the
Memership Roster maintained by the Secretary.

Modifications to this Charter must be passed as amendments by a supermajority
vote of a General Election.

Founders shall always be allowed to vote in a General Election, for
as long as they maintain membership in the organization and do not
claim membership in other organizations with which Cyber@UCR competes.
Founders who would not otherwise meet General Membership criteria
are not counted when considering if a sufficent quorum exists.

\section{Board Elections}

Board Elections are elections closed to organizational officers. Eligible
voters are those officers defined in Article III. These are not to be
confused with elections that determine who sits on the Board; Board
members are confirmed by votes of the General Members (General
Elections).

Passage of organizational Bylaws are conducted by simple majority
of the Board.

Founders shall always be allowed to vote in a Board Election, for
as long as they maintain membership in the organization and do not
claim membership in other organizations with which Cyber@UCR competes.
Founders who would not otherwise be eligible to vote on the Board
are not counted when determining if a sufficent quorum exists.

\part{Ethics}

\section{Ethical Standards}

The Ethics Director shall maintain a Code of Ethics. The Ethics Director
may alone propose changes to the Code of Ethics, which are then passed
by a simple majority in a General Election.

\section{Jurisdiction}

\subsection{By Officers}

The Ethics Director and Ethics Committee investigates violations of
ethics by club officers.

\subsection{By Members of the Ethics Committee}

The President will investigate suspected ethics violations by members
of the Ethics Committee, and assume the duties of the Ethics Director
in that specific case.

\subsection{By General Members, or Other Members Not on the Ethics Committee
or Board}

The Ethics Director shall investigate violations of ethics by other
Members who do not fall into these special categories.

\section{Ethics Hearings and Sanctions}

An Ethics Hearing is called by the Ethics Director in a meeting of
the Board and must be seconded by another Officer. In the case where
a member of the Ethics Committee is implicated, the Ethics Hearing
is called by the President in a meeting of the Board and is seconded
by another Officer.

An Ethics Hearing is presided over by the Ethics Director (or President,
where a member of the Ethics Committee is implicated).

The board for an Ethics Hearing is composed of the Ethics Committee
and at least two Officers. A faculty advisor must be present. This
Ethics Hearing's board may not include a nominee or appointee of the
person implicated.

A faculty advisor must concur with the conclusions and recommendations
of the board of an Ethics Hearing to impose sanctions.

The board for the Ethics Hearing must be unanimous to impose sanctions.

The board must impartially consider the facts of the specific violation
and may impose the following sanctions:
\begin{itemize}
\item Removal of an officer's title or position.
\item Barring the Member from participating in competitions or organizational
projects.
\item Ejection of the Member from the organization for a period of no more
than one year for simple violations.
\item Permanent ejection if the Member is found by local, state, or federal
authorities to be in violation of the law, or in serious violation
of campus policy by campus officials.
\item Referral to state, local, or federal authorities.
\item Written statement of reprimand, to be kept in the organization's record
by the Secretary for at least five years.
\end{itemize}
Members sanctioned by an Ethics Hearing may appeal the decision to
the Board, which may then overturn the Ethics Hearing's findings by
a board supermajority, except where the member was an Officer at
the time of the hearing. Where the sanctioned Member was an Officer,
the appeal must be referred to a General Election and may be overturned
by a supermajority of the general membership. Appeals must be filed
with the Board within seven days.

\section{Duty to Report}

All members have a duty to report all violations of law or ethical
standards of which they have any knowledge to the Ethics Director
or a faculty advisor.

\section{Liability for Ethics Violations}

All members are responsible for their own actions. The tools, techniques,
and procedures discussed in this organization are of a highly sensitive
nature. All members are strictly warned that misuse of these skills
may result in civil or criminal penalties. This organization does
not condone use of techniques that cause systems to perform in unintended
ways without the written consent of the target system's owner. A student
who is considering an activity that might come into conflict with
this organization's core values or the law should consult with the
Ethics Director or a Faculty Advisor before taking action. Any action
in violation of the law or the Code of Ethics is the sole responsibility
of the person who actually performed the action, regardless of who
that person may have consulted.

\part{Administration}

\section{Dues}

The imposition of, or alterations to, financial dues owed by Members
will be agreed upon by a supermajority of the Board.

\section{Budgets and Transactions}

\subsection{Access to Accounts}

Access to accounts controlled by the organization will be granted
to the President, Treasurer, and Faculty Advisor.

\subsection{Approval for Large Transactions}

Transactions in excess of \$30 must be approved in writing by the
President, Treasurer, and Faculty Advisor. Petty transactions less
than this amount may be approved by the President and Treasurer.

\subsection{Shipping of Purchases and Equipment}

Shipping of any equipment purchased by the organization will be to
the UCR Campus, to be entered into inventory by the Treasurer, and
is not to be shipped to the personal residence of any Member, nor
to the address of any outside businesses or organization in which
any Member has a stake or interest.

\subsection{Seperation of Accounts}

Funds offered as grants by the University must be held as seperate
from other funds, and a seperate accounting must be maintained that
describes precisely what such funds were used for with justification.

\section{Dissolution}

In the event of dissolution of the organization, any remaining budget
held by the Organization will be divided equally between IEEE@UCR
and ACM@UCR.

Any physical equipment will be turned over to the Department of Computer
Science and Engineering at UCR for final disposition.

\section{Records}

Records of meetings, the outcome of Ethics Hearings, and the outcome
of all elections will be retained for at least five years by the Secretary.

Records of financial transactions will be held for at least three
years by the Treasurer.

\section{Branding}

The President and Outreach Director will approve all media, logos,
flyers, shirts, graphics, hand-outs, or other materials used by the
organization for branding and advertising.

\subsection{Branding and Sponsorship}

Sponsorship agreements which trigger a fundamental change in branding,
logos, or naming must pass a majority vote of the General Members. 

\section{Bylaws}

A Bylaw is any procedural rule that covers cases not documented in
this Charter, or clarifies how a mandate in this Charter is to be
carried out.

Bylaws must not be in conflict with any rule in this Charter.

A motion to adopt a bylaw may be made by any officer, and must be
seconded by another officer.

A Bylaw is passed by simple majority of the Board.

The Secretary shall maintain a copy of the Bylaws.

\section{Modification of this Charter}

Modifications to this charter require a supermajority vote of the
Board, following which those changes are confirmed by a supermajority
vote of the General Members.

\section{Equipment}

\subsection{Inventory}

The Treasurer shall maintain an inventory of all equipment purchased
by the organization.

Equipment will be inventoried by the Treasurer at the end of each
academic quarter. Each outgoing Treasurer will conduct a joint inventory
with an incoming Treasurer.

\subsection{Acceptable Use}

Equipment owned by the organization is only to be used in activities
sanctioned by the Board. The Board may define ``acceptable use''
for individual pieces of equipment or software held by the organization.

\subsection{Deployment of Equipment}

Deployment of equipment in the UCR Campus will be done in accordance
with existing UCR policy and under the supervision of a concerned
UCR systems group (such as CSE Systems).

\part{Supercession}

Rules in this Charter shall superscede all other rules published by
the Organization as bylaws, executive actions, etc.

Federal, state, and local statues, as well as UCR policy, superscede
rules defined in this Charter.

\part*{Stabilization Clause}

The provisions of this Clause are intended to allow the organization
to react quickly during the volatile period of establishment. This
Clause gives the President and other Founders special powers and latitudes
during the early phase of the organization's establishment to build
a strong and lasting organization.

\section*{Stabilization Clause Section 1: The Establishment Period}

The Establishment Period, for which this Clause is effective, is until
the end of the 2017 Spring Semester at UCR. Before the end of the
Establishment Period, the only legitimate officers recognized under
this Charter will be those elected by the General Members in accordance
with this Charter. Affter the Establishment Period, this Clause expires
and is no longer effective.

\section*{Stabilization Clause Section 2: The Board during Establishment}

Until at least a President, Vice President, Secretary, and Ethics
Director are confirmed, the Board is understood to be all Founders
available to vote. Once those initial Officers are appointed, the
Board is managed normally per the rules of this Charter.

\section*{Stabilization Clause Section 3: Special Powers}

The President may appoint Officers that may act on their mandates
immediately. These Officers must be confirmed by a majority vote of
the Board within two weeks of their appointment. An exception is the
Ethics Director, who must be nominated and confirmed by a simple majority
of the Board.

The Secretary will establish criteria for membership and deliver an
initial Membership Roster to the Board at the earliest convenience.
This Roster is to be updated two weeks before the first General Election,
so that a roster of eligible voters can be established.

The President shall be given the authority to enter into partnerships
and negotiate with faculty, local organizations, companies, and other
people in the interest of securing instructional and material resources
for Cyber@UCR. This includes securing Faculty Advisors.

The Ethics Director may direct that a decision by the President requires
a vote of the Board.

\section*{Stabilization Clause Section 4: Establishment of this Charter and other governing documents}

This Charter was drafted by the first President, Bradley Evans, with
the consent of the Board. It enters into force with supermajority
vote of that Board. Further, an initial Code of Ethics was drafted
by Jerry Jiang, and similarly enters into force with the consent of
the President and a supermajority vote of the Board.

\pagebreak{}

\part*{Addendum I: Founders}

Be it known that the following Members are called Founders, and were
the first to join and help build the organization: 
\begin{itemize}
\item Bradley Evans (Founding President)
\item Braddley Carrey
\item Calvin-Khang Ta
\item Ho-Ren Kang
\item Montana Esguerra
\item Alan Quach (Founding Graduate Advisor)
\item Kevin Dinh
\item Marco Tobon
\item Christopher Yee
\item Jerry Jiang (Founding Ethics Director)
\item Patrick Le
\end{itemize}
And we also acknowledge our founding Faculty Advisors: 
\begin{itemize}
\item Prof. Zhiyun Qian (Faculty Advisor) 
\item Victor Hill (Faculty Advisor) 
\item Nick Turley (Staff Advisor) 
\end{itemize}

\end{document}
